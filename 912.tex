\documentclass{agregfiche}

\title{Leçon 912 - Fonctions récursives primitives et non primitives. Exemples}


% V1 par Charlie Jacomme
% Relu par Jean Goubaut-Larrecq

\begin{document}
\maketitle

\secrapports
\begin{rapport}{2018}
	Il s’agit de présenter un modèle de calcul : les fonctions récursives. S’il est bien sûr important de faire
	le lien avec d’autres modèles de calcul, par exemple les machines de Turing, la leçon doit traiter des
	spécificités de l’approche. Le candidat doit motiver l’intérêt de ces classes de fonctions sur les entiers et
	pourra aborder la hiérarchie des fonctions récursives primitives. Enfin, la variété des exemples proposés
	sera appréciée.
\end{rapport}

\secindispensables

\begin{itemize}
	\item Des \textbf{exemples}, partout, pour tout. (somme finie, produit fini, factorielle, conditionnelles...)
	\item Définitions des fonctions primitives récursives de bases, primitives récursives et non-primitives récursives ($\mu$-récursives, i.e avec minimisation non bornée).
	\item Prédicats primitifs et fonctions caractéristiques. Minimisation bornée.

\end{itemize}

\secasavoir

\begin{itemize}
	\item Fonction d'\bsc{Ackermann}.
	\item Lien avec les machines de \bsc{Turing}, ensembles récursifs, récursivement énumérables et fonctions partielles.
\end{itemize}


\secidees

\begin{itemize}
       \item Lien avec le $\lambda$-calcul.
       \item Élimination de la récursion primitive.
       \item Hiérarchie de \bsc{Grzegorczyk} (si vous savez le motiver de manière convaincante)
       \item Théorème d'itération ($s_{mn}$), borderline.
\end{itemize}

\secpieges

\begin{itemize}
    \item Bien réfléchir et insister sur les motivations.
    \item Mettre des exemples intéressants.
    \item Faire des liens avec les constructions des langages de programmation.
    \item Ne pas définir les fonctions récursives primitives autrement que à partir de celle de bases.
    \item Ne pas se tromper sur la définition de la minimisation. Le bug standard est de seulement dire que sur $f$, le but est de retourner le premier $n$ tel que $f(n) \neq 0$. Avec cette définition, on peux résoudre le problème de l'arrêt... Il faut aussi préciser que l'on renvoie ce $n$ uniquement si la fonction répond (et donc termine) sur les entrées précédentes.
    \item Utiliser des exemples pour souligner le pouvoir expressif, et son évolution.
\end{itemize}

\secquestionsclassiques

\begin{itemize}
	\item Pourquoi considérer les fonctions récursive primitives ? % raison historique avec leur introduction par Gödel pour son premier théorème d'incomplétude. Elles correspondent assez bien à la définition de la récursion dans les langages de programmation, et permettent de bien mettre en évidence la différence d'expressivité entre FOR et WHILE (ce qui est moins clair sur les TM). Enfin, c'est naturel pour tout ce qui est calcul sur les entiers, qui permettent cela dit d'encoder beaucoup de choses.
	\item Différence entre la récursion définie ici et celle des langages de programmation.
        \item Quelle construction capture la boucle FOR ? La boucle WHILE
          ?
          % minimisation borné ou non
	\item L'encodage d'un problème est-il toujours calculable ?
          Pourquoi sont définis ainsi les fonctions récursives primitives
          de bases ?
          % si on encode les machines de turing via (M,b) où b vaut 1 si la
          % machine s'arrete, on a des problèmes...
	\item Exemple d'argument diagonal.
	\item Montrer que telle fonction est primitive récursive.
        \item Calculer les premières valeurs de la fonction d'Ackermann.
        \item À quoi sert la fonction d'Ackermann ? Exemple d'utilisation de sa réciproque.
          % Hierarchie des fonctions PR, complexité Union-Find.
        \item Quelle est la classe de machines de \bsc{Turing} équivalent aux fonctions récursives primitives ? % les machines en temps primitifs récursifs. Il suffit de mettre un timeout. Tout ce qui est calculable avec une complexité pas trop grande est en fait récursif primitif via time out !
        \item Pour tous problèmes ayant une complexité raisonnable,
          est-ce que le FOR suffit pour les décider ?
          % cf au dessus, oui
        \item Si on remplace la récursion primitive par [Insérer une autre construction, par exemple produits et sommes bornées] est-ce qu'on capture toutes les fonctions récursives ?
          % Voir les fonctions élémentaires.
\end{itemize}

\secreferences

\begin{itemize}
\item \reference{Wol}{Introduction à la calculabilité : cours et exercices corrigés}{\bsc{Wolper}}{à la BU/LSV}{Appréciable pour sa pédagogie, et la compréhension des concepts majeurs.}  

\item \input{refs/carton}
\end{itemize}

\secdev

\begin{itemize}
    \item[++]  \dev{Montre que X est ou n'est pas récursive primitive}{[Cori], [Car]}{}{912}{X pour \bsc{Ackermann}, avec une preuve un peu compliqué dans le Cori, ou X = isPrime dans le Carton. Attention, \bsc{Ackermann} est casse-geule, il y a plein de lemmes auxilliaires souvent oubliés (e.g, monotonicté de la fonction considérée). % Validé par le jury en 2019.
}

    \item[+] \dev{Une fonction \bsc{Turing} calculable est $\mu$-recursive.}{[Wol], [Car]}{}{912,913}{Preuve précise mais non pédagogique dans le Cori, claire mais non précise dans le Wolper, précise mais pas finie dans le Carton ...}

    \item[+] \input{dev/lambdaequivrecursive}
\end{itemize}


\end{document}