\documentclass{agregfiche}

% V1 par Gaetan Doueneau
% Relu par Gaetean Doueneau, Aliaume Lopez, Emilie Grienberger, Charlie Jacomme
% Relu par Thomas Chatain

\title{Leçon 903 -- Exemples d'algorithmes de tri. Correction et complexité.}

\begin{document}

\maketitle

\secrapports

\begin{rapport}{2018}

	Sur un thème aussi classique, le jury attend des candidats la plus grande précision et la plus grande rigueur.
	Ainsi, sur l'exemple du tri rapide, il est attendu du candidat qu'il sache décrire avec soin l'algorithme de partition et en prouver la correction en exhibant un invariant adapté. L'évaluation des complexités dans le cas le pire et en moyenne devra être menée avec rigueur : si on utilise le langage des probabilités, il importe que le candidat sache sur quel espace probabilisé il travaille.
	On attend également du candidat qu'il évoque la question du tri en place, des tris stables, des tris externes ainsi que la représentation en machine des collections triées.

\end{rapport}

\begin{rapport}{2017}
    [idem]
    Le jury ne manquera pas de demander au candidat des applications non triviales du tri.
\end{rapport}

\secindispensables

\begin{itemize}
    \item Représentation des données: listes, tableaux.
    \item Propriétés des tris: stable, en place, en ligne.
    \item Algorithmes naïfs: insertion, sélection.
   \item Tri rapide, tri fusion.
\end{itemize}

\secasavoir
\begin{itemize}
    \item Borne inférieure sur les tris par comparaison.
    \item Tri asymptotiquement optimaux: diviser-pour-régner (tri fusion), à base de structures de données (tri par tas).
	\item Analyses de complexité (meilleur, pire, moyenne, probabiliste) et preuves de correction.
	\item Tri linéaires (comptage, base).
\end{itemize}

\secidees

\begin{itemize}
    \item Tri rapide avec médian.
    \item Tri par ABR.
    \item Tim Sort.
    \item Réseaux de tri.
    \item Forcer un tri à être stable.
    \item Tris externe.
\end{itemize}

\secpieges

\begin{itemize}
    \item Il faut être \underline{rigoureux}. Bien définir les opérations élémentaires autorisées.
    \item Pourquoi étudier les tris: intérêt pratique (utiles partout, et déjà codés dans des modules) mais surtout pédagogique (des algorithmes simples pour illustrer les notions de correction, de complexité, et les paradigmes algorithmiques).
    \item Illustrer la complexité par des exemples.
    \item Les algorithmes (qui sont simples) doivent être connus par c\oe ur et sans hésitation.
    \item Analyses probabilistes du tri rapide: faire très attention aux probas.
        \emph{Attention au Cormen}
\end{itemize}


\secquestionsclassiques

\begin{itemize}
    \item Définir un «~tri par comparaison~»~? Pourquoi certains tris sont linéaires
        alors qu'on possède une borne inférieure~?
%     le résultat de l'algorithme ne dépend que de l'arbre de
%comparaison. Les tris sub linéaires utilisent des propriétés en plus
%de
%l'entrée, par exemple dans une plage bornée.

    \item Savez vous quels tris sont utilisés par votre langage de programmation
        favori ?
%         tim sort en python
    \item Quelles sont les complexités des opérations élémentaires que vous
        prenez ?
% multiplication d'entier en O(n^log_2(3))  via karatsuba
    \item Quel tri utiliser en pratique en fonction de la taille des
    données ?
%    radix sort si borné
    \item Donner quelques applications non-triviales du tri ? %
    %recherche par dichotomie, paire d'entiers les plus proches,
    %fréquence la plus élevée dans une liste, sélection du k-ieme
    %élément, paire de points du plan les plus proches, enveloppe
    %convexe
    \item Pourquoi veut-on un tri stable ? Un tri en place ? Un tri
    en ligne ?
%    un tri stable => tri sur des tuples, pour chaque clé à la suite.
%Ex, tri par couleurs puis valeurs de carte, ou radix sort.  un tri
%en
%place = meilleur gestion de la
%mémoire
    \item Comment améliorer le tri rapide pour qu'il soit optimal
    dans tous les cas ?
%     médiane des médianes, inutile en pratique
\end{itemize}

\secreferences

\begin{itemize}
    \item \input{refs/cormen}
    \item \input{refs/beauquier}
\end{itemize}

\secdev

\begin{itemize}
	\item \input{dev/tri_tas}
	\item \dev{Complexité du tri rapide}{[Cor], [Bea]}{}{903,926,931}{Bien faire attention au proba du Cormen, aller voir Beauquier est pertinent. Avoir une idée de l'écart type des performances.}

	\item \input{dev/borne_inf_tri}
	\item \input{dev/tri_bitonique}
\end{itemize}


\end{document}