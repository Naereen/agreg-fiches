\documentclass{agregfiche}

\title{Leçon 909 -- Langages rationnels et automates finis.  Exemples et applications.}

% V1 par Charlie Jacomme
% Relu par Gaetean Doueneau, Aliaume Lopez, Emilie Grienberger, Charlie Jacomme
% Relu par Sylvain Schmitz


\begin{document}

\maketitle

\secrapports

\begin{rapport}{2018}
Pour cette leçon très classique, il importe de ne pas oublier de donner exemples et applications, ainsi
que le demande l’intitulé.
Une approche algorithmique doit être privilégiée dans la présentation des résultats classiques (déterminisation, théorème de
Kleene, etc.) qui pourra utilement être illustrée par des exemples. Le jury pourra naturellement poser des questions telles que : connaissez-vous un algorithme pour décider de l’égalité des langages reconnus par deux automates ? quelle est sa complexité ?
Des applications dans le domaine de l’analyse lexicale et de la compilation entrent naturellement dans
le cadre de cette leçon.
\end{rapport}

\secindispensables

\begin{itemize}
\item  Définitions des automates finis et de la notion de langages reconnaissables. Exemples de languages reconnaissables.
\item Définitions des expressions rationnelles et langages rationnels.
\item Théorème de \bsc{Kleene}, connaître/mentionner les constructions pour la preuve (parmi \bsc{Thompson}, \bsc{Glushkov}, \bsc{Antimirov}, \bsc{McNaughton-Yamada}, \bsc{Brzozoswky-McCluskey}). Attention toutefois, Glushkov qui paraît simple n'est pas facilement trouvable dans les livres.
\end{itemize}

\secasavoir

\begin{itemize}
	\item Existence de langages non reconnaissables. Lemme de pompage. Exemples.
	\item Applications, par exemple~: la compilation, la recherche de motif, le
	model checking (et autres).
	\item Accessibilité, Complétude, déterminisation, minimisation (\bsc{Nerode}).
	\item Propriétés de clôture des langages reconnaissables, et construction
	effective sur les automates.
	\item Les complexités des algorithmes.
  	\item Analyse lexicale et chaîne de compilation.
\end{itemize}


\secidees
% to complete
\begin{itemize}
	\item Des questions de complexité.
        \item Lemme d'\bsc{Arden}.
	\item Divers algorithmes de minimisation (dur à faire en développement)
	\item Reconnaissance par monoïdes.
	\item Lien avec les autres modèles de calcul (Boustrophédon, Machine de \bsc{Turing}.
          qui n'écrivent pas sur leur entrée).
	\item Lien avec la logique (MSO).
\end{itemize}


\secpieges

\begin{itemize}
\item Si l'analyse lexicale et la compilation peuvent être un exemple d'application, ils ne doivent pas prendre une place prépondérante. Les automates à piles ne sont notamment pas au programme. On ne rentrera pas dans de nombreuses définitions pour les grammaires.
\item Cette leçon est basique et fondamentale, il faut très bien couvrir toutes les bases, et éviter de partir trop loin théoriquement.
\item Réfléchir à la complexité.
\item Éviter  \bsc{McNaughton-Yamada}, qui est infaisable sur un exemple.
\item Les expressions rationnelles sont des termes, pas des mots.
\end{itemize}




\secquestionsclassiques
\begin{itemize}
\item Comment tester le vide d'un automate ? Commenter tester
l'acceptation d'un mot ? L'inclusion de langages ? L'égalité de
langages ? Obtenir le complémentaire d'un language ? Complexités ?
(attention aux inputs, regexp vs. $\mathcal{A}$ déterministe vs.
$\mathcal{A}$ non déterministe)

% calcul d'un  automate émondé = polynomial, exécuter l'automate
%sur le mot = O(|A|·|w|) pour non det et O(|A|+ |w|) pour det,
% a inclusion b = a inter comp(b) = vide = pspace complet,  automate
%minimaux égaux = exp car determinisation,
%inverser état finaux, Y a-t-il un AFD complet avec un nombre d'états <= k
%reconnaissant le même langage que A AFN donné en entrée ? = NP-Complet.

\item Dur : Tester l'universalité d'une expression rationnel ?
% p-space complet
\item Automate dont le determinisé est en $2^{|Q|}$ ?
% (a+b)*a(a+b)^n
\item Quelle classe de complexité est reconnu par les automates ?
% espace constant
\item Comment prouver la minimalité d'un automate ?
% Calcul des résiduels
\item Prouver que tel langage $X$ n'est pas régulier.
%Lemme de pompage
\item Coût d'élimination des $\epsilon$-transitions ?
% cloture de relation reflexive transitive, O(Q^2)
\item Avantages/inconvénients des constructions du théorème de
\bsc{Kleene} ?

% Thompson => O(|E|^2)
% Glushkov => exactement |E| +1 état et au plus O(|E|^2) transitions
% dérivé d'antimirov -=> au plus |E| +1 état
% MC Naugthon yamada => proche de Floyd Warshall, dur à dérouler sur
%un exemple
% BRZOZOWSKIet MCCLUSKEY => plus intuitive, plus aisé en pratique

\end{itemize}

\secreferences
\begin{itemize}
\item \input{refs/carton}
\item \input{refs/beauquier}
\item \input{refs/sakarovitch}
\end{itemize}

\secdev
%% sketchy, to complete
\begin{itemize}
\item \input{dev/kleene}
\item \input{dev/presburger}
\item \input{dev/autom_occur}
\item \input{dev/nerode}
\item \input{dev/lemetoile}
\item \dev{Automate d'Aho-Corasick}{[Cro2], [Bea]}{}{907,909,921}{Généralise KMP à plusieurs motifs. [Bea] donne bien mieux les intuitions que [Cro2]}


\end{itemize}


\end{document}
