
\documentclass{agregfiche}

\title{Leçon 928 - Problèmes NP-complets : exemples et réductions}

\begin{document}
\maketitle

\secrapports
\begin{rapport}{2018}
    L’objectif ne doit pas être de dresser un catalogue le plus
    exhaustif possible ; en revanche, pour chaque
    exemple, il est attendu que le candidat puisse au moins expliquer
    clairement le problème considéré, et
    indiquer de quel autre problème une réduction permet de prouver
    sa NP-complétude.
    Les exemples de réduction polynomiale seront autant que possible
    choisis dans des domaines variés :
    graphes, arithmétique, logique, etc. Si les dessins sont les
    bienvenus lors du développement, le jury
    attend une définition claire et concise de la fonction associant,
    à toute instance du premier problème,
    une instance du second ainsi que la preuve rigoureuse que cette
    fonction permet la réduction choisie
    et que les candidats sachent préciser comment sont représentées
    les données.
    Un exemple de problème NP-complet dans sa généralité qui devient
    P si on contraint davantage les
    hypothèses pourra être présenté, ou encore un algorithme P
    approximant un problème NP-complet.
\end{rapport}

\secindispensables

\begin{itemize}
	\item \P\ et \NP.
    \item Réduction polynomiale, complétude.
    \item Exemples de problèmes et de réductions. Théorème de Cook.
\end{itemize}

\secasavoir

\begin{itemize}
    \item Acceptation par certificat.
    \item 3-SAT, 2-SAT, HORNSAT. Mise en forme normal (CNF, DNF).
    \item INDSET, COVER, Cycle Hamiltoniens, Voyageur de commerce.
    \item Somme sous ensemble, programmation entière.

\end{itemize}

\secidees

\begin{itemize}
	\item Algorithme d'approximation (par exemple sur le voyageur du
    commerce).
    \item Réduction d'un problème \NP\ devenant \P.
    \item SAT-solver.
\end{itemize}

\secpieges

\begin{itemize}
	\item Il \textbf{faut} être précis sur la définition de \NP\ : ce
	n'est
	pas juste les machines non déterministes fonctionnant en temps
	polynomial. Pour une
	entrée, est-ce que la TM doit terminer pour tout chemin ? Est-ce
	qu'il faut juste qu'il existe un chemin qui termine ? Est-ce
	qu'il faut juste qu'il existe un chemin acceptant qui termine ?
    \item C'est dur d'éviter le côté catalogue. Essayer de varier les
    domaines, d'avoir des dessins et des exemples de problèmes. Le
    plus possible, donner précisément la définition des problèmes.
    \item Dans le développement, il faut absolument avoir des
    dessins \underline{et} que tout soit formel.
\end{itemize}

\secquestionsclassiques

\begin{itemize}
	\item Quelle est l'intuition derrière \NP?
    \item Pourquoi \P=\NP est une question centrale ?
    \item En quoi les problèmes \NP sont difficiles ?
    \item Telle et telle définition d'acceptation pour une machine
    de \bsc{Turing} non déterministe sont-elles équivalentes ?
    \item Que se passe-t-il si on considère des réductions polynomiales
    non déterministes ?
    \item Que se passe-t-il si on considère des réductions en espace
    logarithmique ?
    \item Justifier que \NP\ est close par réduction polynomiale.
    \item Connaissez-vous un SAT solver ? Utilisé en pratique pour
    quoi ?
    \item Pouvez-vous donner l'idée/les détails de cette réduction ?
    \item Montrer que
\[
            L = \left\{ (M, x, 1^t) ~|~ \langle M, x \rangle \text{
            termine en
                temps  }\leq t, \text{pour }M \text{ non déterministe}
            \right\}
\]
est \NP-complet.

\end{itemize}

\secreferences

\begin{itemize}
    \item \reference{Per}{Complexité algorithmique}{\bsc{Perifel}}{BU}{La référence parfaite pour la complexité. Formel, clair. C'est essentiellement une traduction formelle du \bsc{Arora}.}

\item \input{refs/carton}
\textit{La $NP$-complétude de HamPATH est fausse dedans, regarder le \bsc{Kleinberg}}
\item \input{refs/arorabarak}
\item Éviter le \bsc{Papadimitriou}, qui contient de très nombreuses
  fausses preuves.
\item \reference{Kle}{Algorithm design}{Jon \bsc{Kleinberg}, Éva \bsc{Tardos}}{à la BU}{Bonne réference pour les paradigmes de programmation, très bonne référence pour les algorithmes d'approximation, et les variantes de problèmes NP complets qui deviennent P.}
\end{itemize}

\secdev

\begin{itemize}
\item \input{dev/cook}
\item Preuve de \NP-complétude d'un problème au choix.
\end{itemize}


\end{document}