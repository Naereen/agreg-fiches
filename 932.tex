
\documentclass{agregfiche}

\title{Leçon 932 - Fondements des bases de données relationnelles}

\begin{document}
\maketitle

\secrapports
\begin{rapport}{2018}
    Le cœur de cette nouvelle leçon concerne les fondements
    théoriques des bases de données relationnelles :
    présentation du modèle relationnel, approches logique et
    algébrique des langages de requêtes, liens entre
    ces deux approches.
    Le candidat pourra ensuite orienter la leçon et proposer des
    développements dans des directions diverses : complexité de
    l’évaluation des requêtes, expressivité
    des langages de requête, requêtes récursives, contraintes
    d’intégrité, aspects concernant la conception
    et l’implémentation, optimisation de
    requêtes...
\end{rapport}

\secindispensables

\begin{itemize}
    \item  modèle relationnel, algèbre relationnelle, calcul
      relationnel.
    \item théorème de \bsc{Codd}.
\end{itemize}

\secasavoir

\begin{itemize}
    \item Calcul conjonctif.
    \item Indécidabilité de la satisfiabilité (\bsc{Trakhtenbrot}).
      De l'indépendance de domaine.
\end{itemize}

\secidees

\begin{itemize}
    \item Minimisation.
    \item Expressivité, limites et extensions.
    \item Dépendances fonctionnelles et contraintes d'intégrité.
	Système d'\bsc{Armstrong}.
    \item Implémentations (B-arbres).
    \item Complexité.
\end{itemize}

\secpieges

\begin{itemize}
    \item Il ne suffit pas d'écrire les définitions. Il faut
	comprendre leur liens avec les objets manipulé en pratique.
    \item Donner des exemples de BDD, de requêtes, de résultats.
    \item Pour arriver au théorème de \bsc{Codd}, il faut beaucoup de
    définitions, essayez d'alléger le tout avec des exemples, des
    remarques, des dessins.
\end{itemize}

\secquestionsclassiques

\begin{itemize}
    \item Quelle est l'intérêt des bases de données relationnelles ?
    \item Donner un exemple de requête non domaine indépendante.
	Donner deux BDD qui l'illustre.
    \item Est-ce que dans le modèle relationnel on peut avoir des
    duplicats ? Dans SQL ?
    \item Connaissez vous le principe des clés primaires ?
    \item Pouvez vous écrire la requête permettant de ...?
    \item Quelle est le résultat de la construction du théorème de
    \bsc{Codd} sur tel exemple ?
    \item D'où vient l'indécidabilité de la satisfiabilité ?
    \item SQL est-il plus expressif ? Si oui, donner des exemples de
    constructions supplémentaire.
    \item Quelle est l'utilité des contraintes fonctionnelles ?
\end{itemize}

\secreferences

\begin{itemize}
\item \reference{Abi}{Foundations of databases}{Serge \bsc{Abiteboul}, Richard \bsc{Hull}, Victor \bsc{Vianu}}{LSV}{Bien prendre la version en anglais, la traduction française possédant pas mal d'erreurs. La référence pour la leçon bases de données.}  

\item \reference{Ull}{Principles of database systems}{Jeffrey D. \bsc{Ullman}}{BU}{En anglais. Référence un peu vieille mais évoque tout ce qui était connu en 1982 sur les fondements théoriques des bases de données. Largement suffisant pour l'agrégation.}

\end{itemize}

\secdev

\begin{itemize}
\item \dev{L'inclusion de requêtes conjonctives est NP-Complet}{[Abi]}{}{928, 932}{Réduction originale que le jury n'aura pas forcément l'occasion d'entendre souvent dans la leçon 928. Attention cependant, certains détails sont laissés au lecteurs dans [Abi]}

\item \dev{Correction et complétude du système d'Armstrong}{[Ull]}{}{932}{Développement très simple et parfaitement dans le thème. Savoir le présenter sur un exemple. Attention, il faut supposer qu'il existe au moins deux éléments dans le domaine. Par ailleurs, il arrive tard dans le plan donc bien être sûr de pouvoir le mettre dans les 3 pages.}

\item \dev{Analyse amortie dans les arbres 2-4}{[Bea]}{}{901,921,926,(932?)}{Développement un peu plus original que les arbres AVL, les B-abres étant utilisés en pratique dans postgresql pour faire des indexes de bases de données. Dessins et exemples bienvenue.}

\end{itemize}


\end{document}