\documentclass{agregfiche}

\title{Leçon 907 -- Algorithmique du texte. Exemples et applications.}

% V1 par Aliaume Lopez
% Relu par Gaetean Doueneau, Aliaume Lopez, Emilie Grienberger, Charlie Jacomme
% Relu (rapidement) par Sylvain Schmitz

\begin{document}

\maketitle

\secrapports

\begin{rapport}{2017}

    Cette leçon devrait permettre au candidat de présenter une grande variété d’algorithmes et de paradigmes de programmation, et ne devrait pas se limiter au seul problème de la recherche d’un motif dans un texte, surtout si le candidat ne sait présenter que la méthode naïve. De même, des structures de données plus riches que les tableaux de caractères peuvent montrer leur utilité dans certains algorithmes, qu’il s’agisse d’automates ou d’arbres par exemple. Cependant, cette leçon ne doit pas être confondue avec la 909, «Langages rationnels et Automates finis. Exemples et applications.». La compression de texte peut faire partie de cette leçon si les algorithmes présentés contiennent effectivement des opérations comme les comparaisons de chaînes : la compression LZW, par exemple, est plus pertinente dans cette leçon que la compression de Huffman.

\end{rapport}

\secindispensables

\begin{itemize}
    \item Les algorithmes naïfs de recherche de motif et la complexité.
    \item Les algorithmes non naïfs de KMP et \bsc{Boyer-Moore}.
    \item Recherche de motif par automate.

\end{itemize}

\secasavoir
\begin{itemize}

	\item PLSC, distance d'édition (de \bsc{Levenshtein}).
	\item Fonctions bordures, notions élémentaires de l'algorithmique du texte (préfixe, suffixe, table des périodes).
	\item Exemples de mots où les pire cas sont atteints.
\end{itemize}

\secidees

\begin{itemize}
	\item L'algorithme de \bsc{Karp-Rabin}
	\item Automate de \bsc{Simon}
	\item Regarder KMP sous l'angle des structures de données
	ou de l'algorithme (automate de \bsc{Simon})
	\item
	On peut aussi parler de compression de texte et codes
	correcteurs, mais bien attention à rester dans la leçon.
	\item
	Parler des structures (automates, \bsc{Aho-Corasick}, \bsc{Simon},
	Suffix Tree, Suffix Trie, Suffix Array, Prefix Trie).

	\item Recherche approximative de motif.
\end{itemize}

\secpieges

\begin{itemize}
    \item Bien préciser quels sont les problèmes qu'on cherche
        à résoudre, les entrées, et les complexités des algorithmes.

    \item La taille de l'alphabet est un paramètre de la complexité.

    \item Les dessins, c'est bien.

    \item Organiser de manière pédagogique les algorithmes.
        Cela peut vouloir dire par ``problème", par ``méthode", ou
        bien par ``complexité".

    \item L'analyse lexicale et syntaxique doit être évitée, bien que techniquement
        relevant de l'algorithmique du texte.

\end{itemize}

\secquestionsclassiques

\begin{itemize}
    \item Quel algorithme est utilisé par Grep ?
      % Aho-corasick
    \item Faire la table de bordure de tel motif.
    \item Calculer l'automate des occurrences et le faire tourner sur un exemple.
    \item Donner un exemple de pire des cas pour les différents algorithmes.
    \item Pourquoi est-ce que l'automate de Simon possède un nombre linéaire
      d'arc retours ? Est-il minimal ?
    \item Quel algorithme vaut-il mieux utiliser pour rechercher
      plusieurs motifs dans un même texte ?
      % karp-rabin ou Aho-Corasik
\end{itemize}

\secreferences

\begin{itemize}
    \item \input{refs/crochemore}
    \item \input{refs/stringology}
    \item \input{refs/handbook}
    \item \input{refs/beauquier}
\end{itemize}

\secdev

\begin{itemize}
    \item \input{dev/autom_occur}
    \item \input{dev/plsc}
    \item \dev{Automate d'Aho-Corasick}{[Cro2], [Bea]}{}{907,909,921}{Généralise KMP à plusieurs motifs. [Bea] donne bien mieux les intuitions que [Cro2]}

    \item \dev{Calcul de la distance d'édition}{[Cro2]}{}{907,931}{Bien prendre un coût de $1$ pour chaque opération, quitte à généraliser si le jury pose une question.}

%    \item \temporary{CYK}
%    \item \temporary{KMP}
\end{itemize}


\end{document}
